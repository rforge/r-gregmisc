\documentclass[12pt]{article}
\usepackage{geometry}
\usepackage{url}

%%% Hyperlinks for ``PDF Latex'' :
\ifx\pdfoutput\undefined%%--- usual ``latex'' :
  %% Stuff w/out hyperref
\else%%---------------------- `` pdflatex '' : -- still gives funny errors
  \RequirePackage{hyperref}
  %% The following is R's share/texmf/hyperref.cfg :
  %% Stuff __with__ hyperref :
  \hypersetup{%
    %default: hyperindex,%
    colorlinks,%
    %default: pagebackref,%
    linktocpage,%
    %%plainpages=false,%
    linkcolor=Green,%
    citecolor=Blue,%
    urlcolor=Red,%
    pdfstartview=Fit,%
    pdfview={XYZ null null null}%
    }
  \RequirePackage{color}
  \definecolor{Blue}{rgb}{0,0,0.8}
  \definecolor{Green}{rgb}{0.1,0.75,0.1}
  \definecolor{Red}{rgb}{0.7,0,0}
  %% ESS JCGS v2 :
  %%\hypersetup{backref,colorlinks=true,pagebackref=true,hyperindex=true}
  %%\hypersetup{backref,colorlinks=false,pagebackref=true,hyperindex=true}
\fi

\title{``RForge'' Software Development Site for R packages\\
	~ \\
       \textbf{DRAFT} $ $Revision$ $ }
\date{ $ $Date$ $ }
\author{Gregory R. Warnes\\
  Associate Professor\\
  Department of Biostatistics and Computational Biology \\
  University of Rochester}

\begin{document}

\maketitle

\section{Introduction}

The R statistical software environment is increasingly being used
for data analysis within regulated industries (e.g. pharmaceutical
research and development).  A major attraction of the R system is
the relative ease with which add-on packages can be developed and
deployed, as well as the extensive set of packages that are
already available.  (At the time of this writing, more than 400
add-on packages are available from the R-project web site.)  As a
consequence, add-on packages implementing advanced statistical
methods often appear concurrent with--or even before--the
publication of the manuscripts describing the methods themselves.
The combination of these factors has placed R as a premier tool
for statistical computation in many rapidly developing areas,
including bioinformatics and systems biology.

The core R environment, under the guidance of the R Core Team, is
developed using a standard software model supported by software
development tools\footnote{These tools include a source code version
  control system, an issue-tracking system, file release area,
  developer and user mailing lists, and web documentation.}  hosted by
the R Project web sites (\url{http://www.r-project.org}).  This makes it a
relatively straightforward task to validate the use of R itself
for regulatory environments.  In fact, Tony Rossini of Novartis is
leading an effort to provide the necessary documentation to make this
possible.

Unfortunately, the R project does not currently supply equivalent
software development tools for authors of user contributed packages.
This leaves the authors of these packages to use whatever software
development tools they personally have access to.

While the R project requires a contributed package to pass a minimal
set of tests that ensure proper package structure and
consistency\footnote{This set of tests will automatically perform
  additional regression and functionality tests if appropriate code is
  in place.  However, such code is not required.}, there is no
mechanism for categorizing the quality of these add-on packages.  As a
consequence, the quality of add-on R packages is extremely varied and
there is no straightforward mechanism for determining package quality.
This makes it extremely difficult to validate the use of add-on R
packages for use in a regulated environment.

This project proposes to develop an internet portal that provides
add-on package developers with the same types of tools that are
available to the developers of the core R environment, and to
institute a system for clearly categorizing packages according to
their development status.  This will encourage package developers to
utilize a standard software lifecycle, which will increase the overall
quality of contributed R packages, will make it easier to determine
which package may be appropriately used in specific circumstances, and
will ultimately make it possible to validate a set of well-supported
packages for use in regulated environments.


\section{Goals}

This project has three primary goals:

\begin{enumerate}
\item Enable and encourage the use of an appropriate software
  development lifecycle for user-contributed R add-on packages.

\item Reward package authors with publication credit for the
  development of high-quality add-on packages.

\item Provide a set of well-designed packages with a documented
  software lifecycle process that can easily be validated for use in
  regulated environments.

\end{enumerate}

\section{Benefits}

Providing an appropriate set of software lifecycle tools to R packages
developers will:
\begin{enumerate}

\item Encourage the use of appropriate software lifecycle methods to R
  packages
\item Allow users to easily evaluate the development status of
  available packages.
\item Encourage the development of high-quality add on packages
\item Reduce the risk in using add-on R packages
\item Provide a set of packages which can be easily validated for use
  in regulated environments (GxP, FDA Part 11 compliance, etc.)
\end{enumerate}

\section{Mechanism}

Create and support an internet portal for add-on package developers
that provides software lifecycle tools.  These tools will include:
\begin{itemize}
\item version control system (subversion)
\item issue tracking system
\item file release area
\item web page
\item news lists
\end{itemize} for each individual package.

This portal will support categorizing software releases as
\begin{itemize}
\item development,
\item testing,
\item production, or
\item peer-reviewed/validated.
\end{itemize}

Appropriate qualifications for each release classification will be
established.  Examples of such qualifications include:
\begin{itemize}
\item All package releases must pass the 'R CMD check'
  standard R package tests.
\item Test packages must include working examples for each documented
  function.  These will be used for basic regression testing.
\item Production packages must include a 'vignette' describing the
  basic features of the package and show how these features are used
  in a real analysis.
\item Peer-reviewed/validated packages must include a reasonably
  complete set of unit tests, and must be subjected to code review by
  two independent individuals.
\end{itemize}

In order to encourage academic package developers to submit their
packages to the peer-review/validation process, packages that are
accepted into this category will become publications in a well
recognized peer-reviewed journal, probably the \emph{Journal of
Statistical Software} (\url{http://jstatsoft.org}).

In order to obtain a sufficient pool of software reviewers, each
individual submitting a package to the peer review process will be
required to commit to reviewing (or arranging for reviews) of two
other software packages.  Alternatively, individuals may elect to
pay a fee which will be used to remunerate reviewers for timely
effort.

Software reviewers will complete a standard review form developed by
the community which will ensure that best practices are followed by
both software authors and by reviewers.  This form will include a
commentary on the software that will be published alongside the
software itself once all revisions have been completed and the
package has been accepted for publication.  This commentary will
appear under the reviewers names, mirroring standard software
reviews, ensuring that package reviewers also receive appropriate
publication credit for their work.

\section{Resources}

\subsection{Initial Site Creation}

\begin{itemize}
\item Initial Design Specification -- 1 man month

\item Programming -- 3 man months

  This process should be a straightforward extension of existing
  sourceforge-style tools.  One particularly attractive option is
  'GForge', available from \url{http://gforge.org/} and as a Debian package.

  This will include
  \begin{itemize}
  \item installation and configuration of the web server
  \item design of the web interface and project templates
  \item import of existing R packages
  \item integration with the existing R project web site \& tools
  \item security evaluation and system hardening
  \end{itemize}

\item User Acceptance Testing -- 3 months?

\item Train site maintainer -- 3 month?  (concurrent with acceptance testing?)

\end{itemize}

\subsection{Ongoing Maintenance}

\begin{itemize}

\item Web site hosting $\approx$\$500/yr  (guess)

\item Web site administrator and maintainer -- 20hrs/week (probably an
  over-estimate)

  This individual will:

  \begin{enumerate}

  \item Ensure the ongoing operation of the web site, include host and
    site administration (regular backups!).

  \item Provide technical support for package developers.

  \item Perform some ongoing development to improved the available
    tools, site documentation, category standards, etc.

  \item Acting as a gatekeeper for category changes, including
    assignment of reviewers, processing of review responses, etc.

  \end{enumerate}

\end{itemize}


\subsection{Total Resources}

Under the preliminary resource estimates, funding a total of 1
year of full-time-equivalent (FTE) in salary and benefits, plus a
small additional cost for equipment and web hosting ($<$\$20K?), should
be sufficient to establish and maintain the R package development
portal for 1 full year.  Future years should require only 1/2 FTE
plus web hosting costs ($\approx$\$500?).

\section{Funding}


We propose that interested organizations (Pfizer, Insightful,
Novartis, etc.) jointly fund the R Foundation
(\url{http://www.r-project.org/foundation}) to perform the work
required to create and then maintain the R package development
site.

There are a number of reasons for using the R Foundation to
perform this work.  First and most importantly, the R Community
must have confidence in the organization developing and
maintaining the package development portal.  As the R core and key
R developers are the primary members of the R Foundation, this
ensures that they are involved in oversight of the development and
maintenance of the portal.

Second, funding the project through the R Foundation provides a
straightforward mechanism for a multiple organizations to
participate, potentially reducing the cost for individual
participants, while creating a larger total funding pool.

Third, the R Foundation is a logical organization to shepherd the
continued operation of the R package development portal over the
long term.  The continued operational costs can be funded through
annual contributions from a variety of sources, potentially
including government grants.

Fourth, as an Austrian not-for-profit corporation, contributions
are tax-deductible for European organizations.  With an
appropriate filing (see
\url{http://www.r-project.org/foundation/donations.html}),
 this can also be arranged for US organizations.

Finally, the R community would like to see the R Foundation serve
as a resource accumulator for funding a variety of enchantments to
the R software system that are not easily accomplished via the
current contributed time model.  Funding the R package development
portal via the R foundation will demonstrate the utility of this
mechanism.

\end{document}
