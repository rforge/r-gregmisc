\documentclass{bioinfo}
\usepackage{url}
\copyrightyear{2005}
\pubyear{2005}

\begin{document}
\firstpage{1}

\title{Sample Size Estimation for Microarray Experiments}
\author{
  Gregory R Warnes\,$^{\rm a}$\footnote{to whom correspondence should
    be addressed}\t
  and
  Peng Liu\,$^{\rm b}$
}
\address{
  $^{\rm a}$Nonclinical Statistics, Pfizer Global Research and Development,
  Groton, CT 06340 \\
  $^{\rm b}$Department of Biological Statistics and Computational
  Biology, Cornell University, Ithaca, NY 14853
  }

\maketitle

\begin{abstract}

\section{Motivation:}

Microarray technology is widely applied in biomedical and
pharmaceutical research to detect differences in gene expression
levels between treatment groups.  The information provided by
experiments utilizing microarrays can have very high value, provided
sufficient samples are utilized.  However, microarrays are expensive
to process, in terms of both materials and human effort.
Consequently, an appropriate method for determining the effect of
sample size on power is essential for balancing statistical power
against experiment cost. Unfortunately, the same high
dimensionality that makes microarrays so useful also prevents direct
application of traditional sample size calculation methods.

\section{Method:}

We propose a straightforward method for estimating sample size for
microarray experiments. First, standard deviations from existing
control data for the system of interest are used to estimate the
baseline variability of each gene. Next, a standard sample size
calculation (a pooled t-test in our example) is performed separately
for each gene.  Finally the calculated sample sizes are summarized
through a cumulative plot that displays the tradeoff between sample
size and power. Scientists can use this plot to select the best
sample size for their specific application.  While we illustrate the
method using pooled two-sample t-tests. the method can easily be
adapted to any statistical method for which sample-size formulae are
available.

\section{Results:}

We performed a simulation study to evaluate the performance of our
method for estimating the relationship between sample size and power
for 2-sample pooled-variance t-tests.  Our simulation varied the
primary parameters that could affect the accuracy of the proposed
method.  In order to make the simulation as realistic as possible,
we used variance and covariance information obtained from real data.
While the sample size estimation method does not explicitly account
for the interdependencies among genes, we found that its performance
was unaffected by the level of dependency. We also found that the
true proportion of differentially expressed genes had no effect on
the accuracy of the sample size calculations.  As expected, we found
that large differences in variance between treatments or the use of
alternative multiple testing procedure can effect the accuracy of
our method of sample size estimation.  Fortunately, the method can
be easily extended for either or both of these cases.

\section{Availability:}

The \texttt{ssize} R package, available from the Bioconductor
project (\url{http://www.bioconductor.org}), implements the proposed
method.  In addition, all of the codes and data for the described
simulation have been assembled into an R package,
\texttt{ssize.sim}, which is available upon request.

%\section{Contact:} \href{ $^{\rm a}$ $ Gregory_R_Warnes@groton.pfizer.com $ \\
%                          $^{\rm b}$ $PL61@cornell.edu$}
%                          {$ Gregory_R_Warnes@groton.pfizer.com $  \\
%                           $ PL61@cornell.edu $}
\end{abstract}

\section{Introduction}

High-throughput microarray experiments allow the measurement of
expression levels for tens of thousands of genes simultaneously.
These experiments have been used in many disciplines of biological
research, such as neuroscience (\citealp{Mandel03}), pharmacogenomic
research, genetic disease and cancer diagnosis (\citealp{Heller02}).
As a tool for estimating gene expression and single nucleotide
polymorphism (SNP) genotyping, microarrays produce huge amounts of
data which are providing important new insights.

Microarray experiments are rather costly in terms of materials (RNA
sample, reagents, chip, etc), laboratory manpower, and data analysis
effort.  As a consequence, such experiments often employ only
a small number of replicates (2 to 8) (\citealp{Speed03}). In many
cases, the selected sample size may not be adequate to perform
reliable statistical estimation.  Conversely, a given sample size
may be larger than necessary for answering the question at hand.
Either case can result in an enormous waste of resources. It is
critical, therefore, to perform proper experimental design,
including sample size estimation, before carrying out microarray
experiments. Since tens of thousands of variables (gene expressions)
may be measured on each individual chip, it is essential to take into
account multiple testing and dependency among variables when
calculating sample size.  Estimating sample size for large number of
statistical tests adds new questions to those traditionally posed by
sample size calculations, including: Should one use one or multiple
values for power? Should a single value for minimum effect size be
used?  How should error rates be adjusted for multiple-testing?  How
can we account for dependency among the variables?  Further,
traditional methods for sample size calculation apply only to a
single hypothesis test and cannot be directly applied to the
microarray problem.

Possible approaches to estimating sample size for microarray
experiments range between two extremes. At one extreme, sample size
estimation can be performed by constructing a model for the entire
system, including the realistic error structures and
interdependencies among variables. Provided that the model is
appropriate, this choice should generate a highly accurate answer.
However, it can take tremendous effort to find an appropriate and
general model, not to mention the complexity necessary for fitting
such a model. As an example of this approach, Zien et al. (2003)
proposed a hierarchical model including several different sources of
error and suggest heuristic choices for the key parameters in the
model. Unfortunately, the model itself is non-identifiable,
preventing the use of historical data to directly estimate
the necessary parameters.  Since the behavior of microarray data from
different microarray technologies and for different biological
systems is extremely varied, the applicability of this model and its
heuristic parameters to a particular problem is uncertain.

At the other extreme, standard sample size calculation methods can
be applied individually to each gene, and a summary can be
constructed to provide an overall estimate of the required sample
size.  With this choice, the calculations are simple to set up, the
results can be obtained quickly, and it is easy to incorporate
available data. Further, such methods are easy to understand and to
explain to applied scientists.  However, this approach might not
capture enough of the structure of the system, leading to incorrect
estimates for sample size.  We have applied this simple approach,
and describe a simulation study demonstrating that our method
functions well despite its simplicity.

Hwang et al. (2002) proposed a method that lies between the two
extremes.  This method first identifies differentially expressed
genes and then calculates the power and sample size on a space
reduced by Fisher discriminant analysis. It thus combines the
analysis of data on the current experiment with estimation of sample
size for the next experiment. However, this method does not apply
when preliminary results and a corresponding knowledge of
differentially expressed genes for the specific question of interest
are not available, limiting the general applicability of the
method.

The paper is organized as follows. Section 2 describes our proposed
method in detail. Simulation studies to check the performance of the
method under a variety of conditions are outlined in Section 3.
Section 4 presents results and observations from the simulation
study.  Section 5 provides discussion of the simulation results and
comparison of our method with other proposed methods.

\section{Method}
To illustrate our method, we assume that a microarray experiment is
set up to compare gene expressions between one treatment group and
one control group. We further assume that microarray data has been
normalized and transformed so that the data for each gene is
sufficiently close to a normal distribution that a standard 2-sample
pooled-variance t-test will reliably detect differentially expressed
genes. The tested hypothesis for each gene is:

\begin{equation}
  H_0: \mu_{T} = \mu_{C}  \nonumber
\end{equation}

versus
\begin{equation}
  H_1: \mu_{T} \neq \mu_{C} \nonumber
\end{equation}
                                %
where $\mu_{T}$ and $\mu_{C}$ are means of gene expressions for
treatment and control group respectively.

The proposed procedure to estimate sample size is:

\begin{enumerate}
\item{Estimate standard deviation ($\sigma$) for each gene based on
    \emph{control samples} from existing studies performed on the
    same biological system.}

\item{Specify values for
    \begin{enumerate}
    \item minimum effect size, $\Delta$, (log of fold-change for
          log-transformed data)
    \item maximum family-wise type I error rate, $\alpha$
    \item desired power, $1 - \beta$.
    \end{enumerate}
  }

\item{Calculate the per-test Type I error rate necessary to control
      the family-wise error rate (FWER) using the Bonferroni correction:}
\begin{equation}
  \alpha_G = \frac{\alpha}{G}
\end{equation}
%
where $G$ is the number of genes on the microarray chip.

\item{Compute sample size separately for each gene according to the
    standard formula for the two-sample t-test with pooled variance:}
  \begin{eqnarray}
    \lefteqn{1-\beta} \nonumber \\
    &= 1-T_{n_1+n_2-2} \left( t_{\alpha_G/2,n_1+n_2-2} | \frac{\Delta}{\sigma \sqrt{\frac{1}{n_1} + \frac{1}{n_2}}}\right) \nonumber \\
    &  ~~+~T_{n_1+n_2-2} \left( -t_{\alpha_G/2,n_1+n_2-2} | \frac{\Delta}{\sigma \sqrt{\frac{1}{n_1} + \frac{1}{n_2}}}\right)
    \label{eq:formula}
  \end{eqnarray}
 %
  where $\mathrm{T}_{d}(\bullet|\theta)$ is the cumulative
  distribution function for non-central t-distribution with $d$ degree
  of freedom and the non-centrality parameter $\theta$.

\item{Summarize the necessary sample size across all genes using a
      cumulative plot of required sample size verses power. An
      example of such a plot is given in Figure \ref{fig:CumNPlot}
      for which we assume equal sample size for the two groups, $n =
      n_1 = n_2$.}

\end{enumerate}

\begin{figure}[h]
  \centerline{\includegraphics*[width=3.2in]{CumPlotP.pdf}}
  \caption[Effect of Sample Size on Power]{
    After estimation for each gene separately, all estimated sample
    sizes are summarized in this cumulative plot. Sample size can then
    be selected by balancing the number of genes achieving the desired
    power against the required sample size.}
  \label{fig:CumNPlot}
\end{figure}

\begin{figure}[h]
  \centerline{\includegraphics*[width=3.2in]{CumPowerPlotP.pdf}}
  \caption[Given Sample Size, Fraction of Genes Achieving a Specified Power]{
    Given sample size, this plot allows visualization of the fraction
    of genes achieving a specified power.}
  \label{fig:CumPowerPlot}
\end{figure}

\begin{figure}[h]
  \centerline{\includegraphics*[width=3.2in]{CumFoldChangePlotP.pdf}}
  \caption[Given Sample Size, Fold Change (Effect Size) Necessary to
    Achieving a Specified Power]{Given sample size, this plot allows
    visualization of the fraction of genes achieving the specified
    power for different fold changes.}
  \label{fig:CumFoldChangePlot}
\end{figure}

On the cumulative plot, for a point with $x$ coordinate $n$, the
$y$ coordinate is the proportion of genes which require a sample
size smaller than or equal to $n$, or equivalently the proportion
of genes with power greater than or equal to the specified power
($1-\beta$) at sample size $n$. This plot allows users to
visualize the relationship between power for all genes and
required sample size in a single display.  A sample size can thus
be selected for a proposed microarray experiment based on
user-defined criterion. For the plot in Figure \ref{fig:CumNPlot},
for example, requiring $70\%$ of genes to achieve the $80\%$ power
yields a sample size of 10.

Similar plots can be generated by fixing the sample size and
varying one of the other parameters, namely, significance level
($\alpha$), power ($1-\beta$), or minimum effect size ($\Delta$). Two
such plots are shown in Figures \ref{fig:CumPowerPlot} and
\ref{fig:CumFoldChangePlot}.

Our method for computing sample size is easily adapted to other
statistical tests for detecting differentially expressed genes. For
example, if several groups of treatments are to be compared using
ANOVA, our method can be modified by replacing the pooled t-test
sample size formula (\ref{eq:formula}) with the corresponding
formula for the F-test. Similarly, formula (\ref{eq:formula}) can be
replaced with a paired t-test sample size formula to allow the use
of a paired t-test.

\section{Simulation Study}

Given the simplifying assumptions used by our proposed method, we
chose to perform a simulation study in order to check its
performance under violations of our assumptions and to determine the
influences of additional factors that could affect the accuracy of
the results.

Many variables could affect the accuracy of our sample size
estimation method. For example, our method assumes independence
between tests. However, biology predicts--and measured gene
expressions demonstrate--strong dependency patterns among genes
(e.g. among co-regulated genes).  Unequal variance between the
control and treatment groups is another potential violation of our
method's assumptions. Finally, the proportion of truly
differentially expressed genes might also influence the accuracy of
sample size estimation.  Table \ref{tb:SimuVariables} summarizes
each of the variables that we included in our simulation study.

\begin{table}\centering
  \caption{Tested Variables in Simulation Study}\ \\
  \begin{tabular}{lcl}
    \hline\hline
    \\
    % after \\: \hline or \cline{col1-col2} \cline{col3-col4} ...
    Description & Variable & Tested Levels \\
    \\
    \hline\hline
    Proportion of genes &          & 100\%, 80\%,  \\
    that are dependent  & $\gamma$ & 50\%,   0\%    \\
    \hline
    Proportion of genes with     &     &  100\%, 50\%,\\
    true differential expression & $a$ &   10\%,  5\% \\
    \hline
    Variance ratio &  &\\
    between two groups & $r$ & 1, 3, 10  \\
    \hline
    Minimum effect size  & &2-fold \\
    for scientific relevance & $\Delta$ & 4-fold \\
    \hline
    Choice of multiple  & & Bonferroni,  \\
    comparison method & m.method & FDR* \\
    \hline\hline\\
  \end{tabular}
  \emph{*:}FDR indicates the Benjamini and Hochberg procedure (1995) to
    control false discovery rate
  \label{tb:SimuVariables}
\end{table}

We performed simulation passes for each combination of the
simulation parameters.  The steps of each pass are outlined in
Figure \ref{fig:SimuFC}.

Each pass starts by generating a ``true'' underlying distribution
by sampling standard deviations (for independent genes) and
covariance matrices (for groups of interdependent genes) from a
reference data set.  Our reference data set contains gene
expression values for smooth muscle cells from a control group of
untreated healthy volunteers processed using Affymetrix U133 chips
and normalized per the Robust Multi-array Average (RMA) method of
Irizarry, \textit{et al.} (2003).  A specified fraction $a$ of the
genes is randomly selected as exhibiting a true change in
expression consequent to the applied treatment. For these genes,
the mean is set to the selected effect size $\Delta$.  For all
other genes, the mean is set to $0$.

We then mimic the process of using our method in practice. Thus, we
first compute the required sample size using one set of control
samples from the underlying distribution, and then perform the
corresponding statistical analysis for each of 985 separate sample
sets from the underlying true distribution.  We elected to use 985
samples sets after calculating that 983 samples are required to
provide 95\% confidence that the power obtained from simulation
result is within a margin of 2.5\% of true power.

For initial sample size estimation (left branch in Figure
\ref{fig:SimuFC}), we generate a set of samples based on the
generated ``true'' underlying distribution for use as the historical
reference population in our sample size estimation method.  We then
compute a standard deviation for each gene. These standard
deviations are then used as to estimate the cumulative sample size
curve for 80\% power.

Once the sample size estimation has been completed, 985 separate
experimental populations are generated (right branch of Figure
\ref{fig:SimuFC}).  The standard two-sample t-test with pooled
variance is then applied to each individual population using a range
of sample sizes.  Error rate within populations is controlled using
one of two different methods, either the Bonferroni method for
family-wise Type I error rate or the Benjamini and Hochberg
procedure (\citealp{Benjamini95}) for False Discovery Rate (FDR).
The tests from all populations are then pooled to determine the true
power for each gene at each sample size. This allows the generation
of the true cumulative sample size by summarizing across genes,
which is then compared with the estimated curve generated by our
proposed method.

\begin{figure}[h]
  \centerline{\includegraphics*[width=3.2in]{SimuFC.pdf}}
  \caption[Flow Chart for Simulation Study]{ Simulation study
  performed to check the performance of proposed method.  One set of
  gene expression sample is generated and calculated for sample size
  following the steps descried in Method. Many sets of gene
  expression data are generated and tested with standard
  pooled-variance t-test. Test results are pooled to get
  the true power for each sample size. Estimated sample size and
  simulation results are compared at the end.}  
  \label{fig:SimuFC}
\end{figure}

\section{Results}

\subsection{Dependency}

Independence of the genes is a convenient statistical assumption,
allowing application of statistical tests separately to each
gene. However, there is considerable biological and statistical
evidence that the expression levels of genes are not independent
(e.g. many transcription factors collaborate to regulate the
expression of any one gene). To check the performance of the
proposed method in estimating sample size when gene expressions are
dependent in groups, we compared estimation of sample size based on
proposed method with different proportions ($\gamma$) of genes that
are in fact interdependent in groups containing up to 50
genes. Figure
\ref{fig:ResDep} presents the result for the two most extreme cases:
all genes are independent of each other (panel (a)) and 100\% of
genes come from interdependent groups (panel (b)). The other
parameters for the two displayed panels are identical: $\alpha =
0.05, 1 - \beta = 80\%, a = 1, r = 1, \Delta = 1$ (2-fold change)
and Bonferroni adjustment for controlling family wise type I
error. In both cases, as well as at intermediate levels of
dependency, the estimated sample size curve overlaps the results
from simulation.  This demonstrates that interdependency among genes
has little or no effect on sample size estimation using our
method. This result is reassuring, given the high degree of
interdependence known to exist among genes.

\begin{figure}[h]
  \centerline{\includegraphics*[width=3.2in]{ResDepF.pdf}} 
  %%% ResDepF_2005.pdf 
  \caption[Effect of interdependency of genes on sample size
    estimation] {Effect of interdependency of genes on sample size
    estimation.  Cumulative plots for sample sizes from both
    estimation and simulation result are generated for the following
    variable values: $a = 1, r = 1, \Delta = 1$ (2-fold change) and
    Bonferroni adjustment for controlling family-wise type I
    error. Dashed green lines are estimated sample sizes while solid
    black lines are sample sizes obtained from simulation result.}
  \label{fig:ResDep}
\end{figure}

\subsection{Proportions of Genes with True Differential Expression}

The proportion of genes with true differential expression will vary
considerably among treatments. We suggest using all available genes
for calculating cumulative plot unless information is available
indicating that a specific subset of genes is more likely to be
truly differentially expressed. In this latter case, this subset or
``focus'' list of genes can be used for sample size estimation
instead of or in addition to performing the calculations for all
genes.

We simulated four proportions of genes with true differential
expression, $a = 100\%, 50\%, 10\%$ and $5\%$. None of the tested
levels showed any effect on the accuracy of sample size
estimation, i.e., the estimated sample size curve overlapped with
true result from simulation for all tested proportions of
differentially expressed genes (Figure \ref{fig:ResAlt}).

\begin{figure}[h]
  \centerline{\includegraphics*[width=3.2in]{ResAlt.pdf}}
  \caption[Effect of the proportion of genes with true differential
    expression on sample size estimation] {Effect of the proportion of
    genes with true differential expression on sample size estimation.
    Cumulative plots for sample sizes from both estimation and
    simulation result are generated for the following variable values:
    $\gamma = 1, r = 1, \Delta = 1$ (2-fold change) and Bonferroni
    adjustment for controlling family-wise Type I error. Dashed green
    lines are estimated sample sizes while solid black lines are
    sample sizes obtained from simulation result.}
  \label{fig:ResAlt}
\end{figure}

\subsection{Variance Ratio}

As described here, our method employs the sample size formula for
the two-sample pooled-variance t-test.  This test
is only valid when the comparison groups have approximately equal
variance. To determine the sensitivity of our method to the equal
variance assumption, we simulated variance ratio between two groups
of 1, 3 or 10.  Since the problem is symmetric, it was unnecessary
to test the ratios 1/3 and 1/10.

As expected, when the equal variance assumption is satisfied
(variance ratio is 1) the estimated sample size agrees exactly
with the true sample size. When the variance ratio is larger than
one, the simulations show that the true sample size increases
while the estimated sample size remains constant (Figure
\ref{fig:ResVar}).   This is as expected and simply confirms that
the two-sample pooled t-test (\ref{eq:formula}) is not robust
against unequal variance.

Thus, when the treatment and control groups are expected to have a
variance ratio greater than 3 or smaller than 1/3, we recommend
replacing the pooled t-test sample size formula (\ref{eq:formula})
with an appropriate unequal variance t-test sample size formula,
and, of course, using the unequal-variance t-test for detecting
differentially expressed genes.

\begin{figure}[h]
  \centerline{\includegraphics*[width=3.2in]{ResVarF.pdf}}
  \caption[Effect of variance ratio between treatment groups on
    sample size estimation] {Effect of variance ratio between
    treatment groups on sample size estimation.  Cumulative plots for
    sample sizes from both estimation and simulation results are
    generated for the following variable values: $\gamma = 1, a = 1,
    \Delta = 1$ (2-fold change) and Bonferroni adjustment for
    controlling family-wise Type I error. Dashed green lines are
    estimated sample sizes while solid black lines are sample
    sizes obtained from simulation result.}  
  \label{fig:ResVar}
\end{figure}

\subsection{Minimum effect Size}

As the minimum effect size ($\Delta$) increases, genes that are
truly differentially expressed are easier to detect and the
required sample size is correspondingly smaller.  In our
simulation, when we require at least 80\% genes to achieve 80\%
power, a per-group sample size of 6 is required to detect 2-fold
changes ($\Delta = 1$), while only 4 samples per group is needed
to detect 4-fold changes ($\Delta = 2$) when $\gamma=1$, $a=1$,
$\text{variance ratio}=1$ and using the Bonferroni correction to
control family-wise type I error. Our simulations demonstrate
that, as expected, change in $\Delta$ has no effect on the
accuracy of our sample size estimation methods since the value of
$\Delta$ is explicitly included in the sample size formula
\ref{eq:formula}.

\subsection{Multiple Comparison Method}

Many different methods have been proposed to control error rate for
multiple testing.  These include the Bonferroni correction for
strong control of family-wise Type I error, the Benjamini and
Hochberg method for controlling false discovery rate (FDR)
(\citealp{Benjamini95}) and Storey's q-value procedure to control
FDR or positive FDR (pFDR) (\citealp{Storey02}).  Both the FDR and
pFDR methods control the proportions of false positives among all
positive findings.  These allow a certain proportion of Type I
errors within the list of positive calls, usually resulting in
higher power than Bonferroni adjustment which attempts to control
the probability of any Type I error.

Although the FDR methods are applied more often in practice, we
chose to utilize the Bonferroni method in sample size estimation for
two reasons: First, the Bonferroni method is simple to apply.
Second, it provides a conservative estimate of power and hence
sample size.  To determine the extent of the conservatism, we
compared the estimated sample size using the Bonferroni correction
with the true sample size needed when controlling FDR via the
Benjamini and Hochberg procedure (\citealp{Benjamini95}). As
expected, Figure \ref{fig:ResMtd} shows that the true sample sizes
needed for the FDR method is considerably smaller than the sample
size estimated using the Bonferroni adjustment.

\begin{figure}[h]
  \centerline{\includegraphics*[width=3.2in]{ResMtd.pdf}}
  \caption[Effect of multiple comparison method on sample size
    estimation] {Effect of multiple comparison method on sample size
    estimation.  Cumulative plots for sample sizes from both
    estimation and simulation result are generated for the following
    variable values: $\gamma = 1, a = 1, r = 1, \Delta = 1$ (2-fold
    change). Dashed green lines are estimated sample sizes while solid
    black lines are for sample sizes obtained from simulation result.}
  \label{fig:ResMtd}
\end{figure}

\section{Discussions}

The number of chips included in microarray experiments directly
affects the reliability of any conclusions from data analysis.
Thus, it is critical to have an appropriate method for selecting the
number of chips required to obtain reliable data in order to avoid
wasting effort and resources. The huge number of correlated outcomes
prevents traditional methods of estimating sample size from being
directly applied to microarray experiments, at the same time that
the increasing frequency of microarray experiments demands that
appropriate methods be developed.

The literature proposes several methods of sample size calculation
for microarray experiments. None are as straightforward and flexible
as the method proposed here. Hwang et al. (2002) proposed a method
that first identifies differentially expressed
genes and then calculates power and sample size on a reduced
parameter space. Dow studied relationship between minimum detection
size and sample size for a specific experiment (2003). Both methods
cannot be applied to cases where preliminary results, including a
knowledge of differentially expressed genes, are not available. Zien
et al.  (2003) proposed a hierarchical model which includes several
different sources of error and recommend heuristic choices for the
key parameters in the model.  After attempting unsuccessfully to fit
this model to real data, we discovered that that the model itself is
non-identifiable.  This makes it inappropriate to employ across the
variety of sample types we see in practice.

We have instead proposed a very straightforward method for
estimating required sample size that is easy to apply and is simple
to adapt or extend. The key component of our method is the
generation of cumulative plot of the proportion of genes achieving a
desired power as a function of sample size, based on simple
gene-by-gene calculations.  While this mechanism can be used to
select a sample size numerically based on pre-specified conditions,
its real utility is as a visual tool for helping clients to
understand the tradeoff between sample size and power.  In our
consulting work, this latter use as a visual tool has been
exceptionally valuable in helping scientific clients to make the
difficult tradeoffs between experiment cost and statistical power.

In order to check the performance of our proposed method, we
performed an extensive simulation study. Of the variables tested in
the simulation (Table \ref{tb:SimuVariables}), only variance
structure and the multiple testing method in had any substantial
impact on the accuracy of the sample size estimation.  Both of these
were expected \textit{a-priori} due to the specific assumptions
employed and are easily corrected by appropriate modifications to
the method (i.e. use of the appropriate sample-size formula and use
of the appropriate multiplicity correction during sample size
estimation, respectively).

Neither the proportion of genes that are interdependent ($\gamma$)
nor the proportion of genes with true differential expression ($a$)
have any meaningful effect on the accuracy of our sample size
estimates.  This is fortunate, since a problem in either area would
be difficult to correct based on information available before the
experiment is run.  That is, while we expect that genes have high
correlation within regulatory and functional groups, it is currently
impossible to determine the level of correlation among sets of genes
for a given experiment before it is run.  Likewise it is difficult
to predict the fraction of genes with true differential expression
before running the experiment.

In all of the cumulative plots of sample size versus number of genes
achieving 80\% power we found that there is a steep increase at
small sample numbers. For example, Figure \ref{fig:ResDep} shows
that an increase of sample size from 4 to 5 assures 40\% more genes
with desired power (from 22\% to 64\%).  Another increase of one
chip (sample size = 6) results in about 80\% of genes with the desired
power. Taking this into account, an ideal sample size might be the
leftmost number at the top of the steep portion of the curve.

Our simulation has also confirmed that the Benjamini and Hochberg
procedure for controlling FDR has much more power than Bonferroni
adjustment (Figure \ref{fig:ResMtd}). A useful extension of proposed
method of would be the use of a FDR control instead of the
conservative Bonferroni adjustment. One approach, suggested by Yang
\textit{et al.} (2003) computes an individual type I error to
control FDR based on an initial guess of how many genes will be
differentially expressed ($n_0$). This could be used to modify our
method to provide a better estimate when a FDR adjustment will be
employed.  Unfortunately, the proposed transformation of FDR to type
I error rate is conservative and may control FDR at a level more
strict than desired. We are currently investigating this and other
methods for applying FDR-control during the sample size estimation
process in the hopes of location a method that controls FDR more
accurately.


\section*{Acknowledgment}

This work was supported by Pfizer Global Research and Development.

\begin{thebibliography}{}

\bibitem[Benjamini and Hochberg, 1995]{Benjamini95} Benjamini, Y.
  and Hochberg, Y. (1995) Controlling the false discovery rate: a
  practical and powerful approach to multiple testing, {\it Journal
  of Royal Statistical Society B}, {\bf 57:1}, 289-300.

\bibitem[Dow, 2003]{Dow0303} Dow,G.S. (2003) Effect of sample size
  and p-value filtering techniques on the detection of
  transcriptional changes induced in rat neuroblastoma (NG108) cells
  by mefloquine, {\it Malaria Journal}, {\bf 2}, 4.

\bibitem[Heller, 2002]{Heller02} Heller, M. J. (2002) {DNA
  microarray technology: devices, systems, and applications}, {\it
  Annual Review in Biomedical Engineering}, {\bf 4}, 129-153.

\bibitem[Hwang {\it et~al}., 2002]{Hwang02} Hwang,D., Schmitt,
  W. A., Stephanopoulos, G., Stephanopoulos, G. (2002) Determination
  of minimum sample size and discriminatory expression patterns in
  microarray data, {\it Bioinformatics}, {\bf 18:9}, 1184-1193.

\bibitem[Irizarry {\it et~al}., 2003]{Irizarry03} Irizarry, R.A.,
  Hobbs, B., Collin, F., Beazer-Barclay, Y.D., Antonellis, K.J.,
  Scherf, U., Speed, T.P. (2003) Exploration, normalization, and
  summaries of high density oligonucleotide array probe level data,
  {\it Biostatistics}, {\bf 4:2}, 249-264.

\bibitem[Mandel {\it et~al}., 2003]{Mandel03} Mandel, S.,  Weinreb,
  O., Youdim, M. B. H. (2003) Using cDNA microarray to assess
  Parkinson's disease models and the effects of neuroprotective
  drugs, {\it TRENDS in Pharmacological Sciences}, {\bf 24:4},
  184-191.

\bibitem[Yang and Speed, 2003]{Speed03} Yang, Y. H., Speed, T.
  {Design and analysis of comparative microarray experiments \it
  Statistical analysis of gene expression microarray data}, {Chapman
  and Hall}, 51.

\bibitem[Storey, 2002]{Storey02} Storey, J., (2002)
  A direct approach to false discovery rates, {\it Journal of Royal
  Statistical Society B}, {\bf 64:3}, 479-498.

\bibitem[Yang {\it et~al.}, 2003]{Yang03} Yang, M. C. K., Yang,
  J. J., McIndoe, R. A., She, J. X. (2003) Microarray experimental
  design: power and sample size considerations, {\it Physiological
  Genomics}, {\bf 16}, 24-28.

\bibitem[Zien {\it et~al.}, 2003]{Zien03} Zien, A., Fluck, J.,
  Zimmer, R., Lengauer, T. (2003) Microarrays: how many do you
  need?, {\it Journal of Computational Biology}, {\bf 10:3-4},
  653-667.

\end{thebibliography}
\end{document}
