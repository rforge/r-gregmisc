\documentclass{asaproc}
\usepackage{inputenc}
\usepackage{graphics}

\usepackage{times} \usepackage{fontenc}

\title{Open Source Software in Pharmaceutical Research}

\author{
  Gregory R. Warnes$^{1,2,3}$, James A. Rogers$^4$ Max Kuhn$^4$ 
\\
  Center for Biodefense Immune Modeling, University of Rochester $^1$ \\
  Department of Biostatistics and Computational Biology, 
  University of Rochester $^2$ \\
  REvolution Computing, Inc. $^3$ \\
  Research Statistics, Pfizer, Inc. $^4$
}

\begin{document}

\maketitle

\begin{abstract}
  
Open-Source statistical software is being used with increasing
frequency for the analysis of pharmaceutical data, particularly in
support of ``omics'' technologies within discovery. While it is
relatively straightforward to employ open-source tools for basic
research, software used in any regulatory context must
meet more rigorous requirements for documentation, training,
software life-cycle management, and technical support.

\vspace{1em}

We will focus on \textsc{R}, a full-featured open-source statistical
software package. We'll briefly outline the benefits it provides, as
seen from the perspective of a discovery statistician, show some
example areas in which it may be used, and then discuss the
documentation, training, and support required for this class of use.

\vspace{1em}

Next we will discuss what is needed for organizations to be
comfortable with employing open-source statistical software for
regulatory use within clinical, safety, or manufacturing. We will
then talk about how well or poorly \textsc{R} meets these requirements,
highlighting current issues.  Finally, we will discuss options for
third-party commercial support for \textsc{R}, and evaluate how well they
meet the requirements for use of \textsc{R} within both regulated and
non-regulated contexts.

\vskip6pt


{{\bf \noindent Keywords}: Text, Text.}


\end{abstract}

\section{Introduction}

  Open-Source statistical software is being used with increasing
  frequency for the analysis of pharmaceutical data, particularly in
  support of ``omics'' technologies within discovery. While it is
  relatively straightforward to employ open-source tools for basic
  research, software used in any regulatory context must meet more
  rigorous requirements for documentation, training, software
  life-cycle management, and technical support.

\section{Requirements}

  Software used in mission critical and regulated contexts must
  exhibit 7 key attributes:

\begin{description}

\item Performs the required tasks

\item Demonstrate that computer output is correct, or at least consistent..

\item Given the same data, the same results can be
  obtained, potentially much later in time.

\item Documentation is available or can easily be
  generated for the entire software life-cycle: Specification, Design,
  Development. Testing, Deployment, Change Management

\item Track everything done to data and the system

\item Doesn't change too fast, so that there is enough time to
  develop required documentation
  
\item Guaranteed (by \$\$) availability of external expense
  for installation, problem resolution, bug fixes, feature
  development, training, application development, consulting

\end{description}

\section{What is \textsc{R}?}

  \begin{itemize}

    \item System for statistical computing and graphics

    \item Language is very similar to the S-Plus
      
    \item Full featured support for statistical and graphical techniques:

      \begin{itemize}
      \item linear and nonlinear modeling,
      \item classical statistical tests,
      \item time-series analysis,
      \item classification,
      \item clustering
      \item ...
      \end{itemize}

    \item Highly extensible with good development tools
      
    \item \emph{Huge} library of user-contributed add-on packages: $>850$ ! 
      
    \item Source code is freely available

\end{itemize}

\section{Status of \textsc{R}}

\begin{description}

\item +++ This is \textsc{R}'s strength. Largely provided by
  the $>850$ user-supplied add-on packages.  \textsc{R} currently provides more
  functionality than any other statistical software system and is
  growing rapidly.

\item --- Most of the functionality of \textsc{R} comes from
  user-developed add-on packages ($>850$!), but there is currently no formal
  mechanism for evaluating the level of quality of these packages (e.g.:
  development, test, production, peer reviewed, validated) or
  documentation that they accomplish the required tasks.
  
\item --- Currently, add on packages do not display
  version information when loaded, making it difficult to know what
  versions were utilized for a given analysis, and thus impossible to
  reliably replicated.

\item --- While the \textsc{R} core team has a well defined and
  managed process for design, development, testing, release, and
  change management, no formal documentation of this process appears
  to exists (aside from the specifications of the language itself).
  No centrally defined or managed process appears to exist for add-on
  packages.

\item --- \textsc{R} has no built-in no audit log, either for data
  analysis steps or for changes to the system (e.g.: package updates,
  patches)
  
\item --- The \textsc{R} core team releases minor (major.minor.patch)
  versions twice a year.  Since bug fixes are currently applied only
  to the latest released version of the system, it is difficult to
  properly support embedded and validated systems where one may need
  to resolve bugs in \textsc{R}, but must constrain the \textsc{R}
  version to remain constant for long periods due to the burden of
  documentation and testing that must be performed.

\item --- While there is an increasingly large pool of
  statisticians and statistical consulting groups that have \textsc{R}
  expertise, no organization formally supports \textsc{R} at this time.

\end{description}

\section{Moving Forward}

\begin{description}
  \item[Functional] Already a strength. Continue!
  \item[Verifiable] \textsc{RForge} proposal
    \begin{enumerate}
      \item Develop a SourceForge-like system for contributed packages: 
%        \begin{itemize}
%        \item version control system (subversion)
%        \item issue tracking system
%        \item file release area
%        \item web page
%        \item news lists
%        \end{itemize} for each individual package.
      \item Support package status categories, including clear standards
        \begin{itemize}
        \item development,
        \item testing,
        \item production, or
        \item peer-reviewed/validated.
        \end{itemize}
      \end{enumerate}
  \item[Repeatable] Display versions of packages on load
  \item[Documentable] ~ 
    \begin{enumerate}
    \item Formally document the development process used for \textsc{R}
    \item Provide tools to perform and document this process for
      add-on packages
    \item Develop validation templates for use by organizations
    \item Encourage commercial vendors to support \textsc{R} and to provide
      additional validation effort and associated documentation.
    \end{enumerate}
  \item[Auditable] Add an audit-log facility
  \item[Stable] Establish a system for back-porting bug fixes to
    previous versions.
  \item[Supported] Encourage commercial vendors to formally support \textsc{R}.
\end{description}

\section{News Flash! \textsc{RPro} from \emph{REvolution Computing} }

\begin{quote}
  2006-08-01 {\bf New Haven, CT}: \emph{REvolution Computing}
  announces the immediate availability of \textsc{RPro}, an enterprise-strength
  statistical computing environment providing the strengths of the
  open source \textsc{R} statistical software system from the R-Project coupled
  with the enterprise-level support and high-performance computing
  expertise of \emph{REvolution Computing}.
\end{quote}

  \vspace{1em}

  Additions to \textsc{R}:
      
      \begin{itemize}
        
      \item Technical Support
        
      \item Simple Installation and Maintenance
        
      \item Performance Tuning
        
      \item Documentation and Training
        
      \item Validation Materials
        
      \item Consulting and Services
        
      \end{itemize}
    
\begin{quote}

  2006-08-01 {\bf New Haven, CT}: \emph{REvolution Computing}
  announces the immediate availability of \textsc{NetWorkSpaces} for
  \textsc{RPro} (\textsc{NWS}). \textsc{NWS} enables calculations to
  be automatically distributed across multiple processors in clusters.
  Distributing the data and/or work across multiple processors permits
  a dramatic decrease in time to completion of large computational
  tasks or permits a dramatic increase in those calculations size,
  length or complexity.  \textsc{NWS} fully supports Microsoft Windows
  Compute Cluster Server 2003 (\textsc{CCS}), which provides a
  security enhanced and affordable high performance computing
  solution.

\end{quote}

  \includegraphics*{MicrosoftWindowsClusterComputeServer2003.pdf}
  \includegraphics*{RevolutionComputingInfo.pdf}

\section{More Information}

\begin{itemize}
  \item Personal:
    \begin{description}
    \item greg@warnes.net
    \item http://www.warnes.net/Research
    \end{description}
    \vspace{1em}

  \item University of Rochester:
    \begin{description}
    \item warnes@bst.rochester.edu
    \item http://www.urmc.rochester.edu/smd/biostat
    \end{description}
    \vspace{1em}
    
  \item REvolution Computing:
    \begin{description}
    \item greg@revolution-computing.com
    \item http://www.revolution-computing.com
    \end{description}
    \vspace{1em}

  \end{itemize}

\section*{References}
\end{document}
{\footnotesize\begin{description}\itemsep=3pt

\item {\em Annual Book of ASTM Standards, Volume 4.10}, Committee D-7,

American Society for Testing and Materials, West Conshohocken, Pennsylvania.

\item Cochran, W.G. (1957),  ``Analysis of Covariance: Its Nature and Uses,''

{\em Biometrics}, {\bf 13}, 261-281.

\item Cox, D.R. (1957),  ``The Use of a Concomitant Variable in Selecting an

Experimental Design,''  {\em Biometrika}, {\bf 44}, 150-158.

\item David, H.A. and Gunnik, J.L. (1997), ``The Paired {\em t} Test

Under Artificial Pairing,'' {\em The American Statistician}, {\bf 51},

9--12.

\item Gibbons, R.D. (1994), {\em Statistical Methods for Groundwater

Monitoring},

John Wiley and Sons, New York.

\item Guttman, I. (1970), {\em Statistical Tolerance Regions:

Classical and Bayesian}, Hafner Publishing Company, Darien,

Connecticut.

\item Michigan DEQ (1994), {\em Guidance Document, Verification of

Soil Remediation}, Hazardous Waste Program Section, Waste Management

Division, Michigan Department of Environmental Quality, Lansing, Michigan.

\item Owen, D.B. (1963), {\em Factors for one-sided tolerance limits

and for variable sampling plans}, Sandia Corporation Monograph No.

SCR-607 (19th edn).

\item {\em MIL-HDBK-17-1, Composite Materials Handbook, Volume 1,

Guidelines for Characterization of Structural Materials}, Department

of Defense Single Stock Point, Philadelphia, Pennsylvania

(http://www.dodssp.daps.mil).

\item Verrill, S.P. (1993),  ``Predictor Sort Sampling, Tight

{\em T}'s, and the Analysis of Covariance,'' {\em Journal of the American

Statistical Association}, {\bf 88}, 119--124.

\item Verrill, S.P. (1999),  ``When Good Confidence Intervals Go Bad:

Predictor Sort Experiments and ANOVA,'' {\em The American

Statistician}, {\bf 53}, 38--42.

\item Verrill, S.P., Herian, V.L., and Green, D.W. (2002a), ``Predictor

Sort Sampling and Confidence Bounds on Quantiles I: Asymptotic Theory,''

under review.

\item Verrill, S.P., Herian, V.L., and Green, D.W. (2002b), ``Predictor

Sort Sampling and Confidence Bounds on Quantiles II: $k$'s for Small

Samples,'' in preparation.

\item Warren, W.G., and Madsen, B. (1977),  ``Computer-Assisted Experimental Design

in Forest Products Research: A Case Study Based on Testing the 

Duration-of-Load Effect,''  {\em Forest Products Journal}, {\bf 27}, 45-50.

\end{description}}

\end{document}

