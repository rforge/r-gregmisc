\documentclass[a4paper]{article}
\usepackage[round]{natbib}
\usepackage{array}
\usepackage{graphicx}
\usepackage{fullpage}
\bibliographystyle{plain}

\begin{document}

\title{Open Source Software in Pharmaceutical Research}

\author{
  Gregory R. Warnes\\
  Pfizer Inc., USA\\
  \texttt{gregory.r.warnes@pfizer.com}
}

\date{2006-02-08}

\maketitle

\abstract{
  
Open-Source statistical software is being used with increasing
frequency for the analysis of pharmaceutical data, particularly in
support of ``omics'' technologies within discovery. While it is
relatively straightforward to employ open-source tools for basic
research, software used in any regulatory context must
meet more rigorous requirements for documentation, training,
software life-cycle management, and technical support.

We will focus on R, a full-featured open-source statistical
software package. We'll briefly outline the benefits it provides, as
seen from the perspective of a discovery statistician, show some
example areas in which it may be used, and then discuss the
documentation, training, and support required for this class of use.

Next we will discuss what is needed for organizations to be
comfortable with employing open-source statistical software for
regulatory use within clinical, safety, or manufacturing. We will
then talk about how well or poorly R meets these requirements,
highlighting current issues.  Finally, we will discuss options for
third-party commercial support for R, and evaluate how well they
meet the requirements for use of R within both regulated and
non-regulated contexts.

}

%\section{Introduction}


%\section{Discussion}

%\section{Conclusion}

%\begin{thebibliography}{99}

%\bibitem{Friendly.1992} ``Graphical Methods for Categorical Data'' Friendly, M.
%  (1992) \textit{Proceedings of SAS SUGI 17 Conference}

%\end{thebibliography}

\end{document}
