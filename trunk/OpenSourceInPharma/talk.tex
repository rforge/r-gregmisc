\documentclass[10pt]{beamer} \usepackage{beamerthemesplit}
\mode<presentation> { \usetheme{Warsaw}
  % or ...

  \setbeamercovered{transparent}
  % or whatever (possibly just delete it)
}
%\setbeameroption{show notes}

\usepackage[english]{babel} \usepackage[latin1]{inputenc}

\usepackage{times} \usepackage[T1]{fontenc}

% Delete this, if you do not want the table of contents to pop up at
% the beginning of each subsection:
%\AtBeginSubsection[]
%{
%  \begin{frame}<beamer>
%    \frametitle{Outline}
%    \tableofcontents[currentsection,currentsubsection]
%  \end{frame}
%}

\setbeamersize{text margin left=0.5cm}
\setbeamersize{text margin right=0.5cm}


\title{Open Source Software in Pharmaceutical Research}

\author{
  Gregory R. Warnes\inst{1} \and 
  James A. Rogers\inst{2} \and
  A. Max Kuhn\inst{2}
}

\institute[U. Rochester, Pfizer]{
  \inst{1}%
  Department of Biostatistics and Computational Biology, University of Rochester \vspace{-5pt}
  \and
  \inst{2}%
  Statistical Applications, Pfizer, Inc. \vspace{-5pt}
}


\date[2006 JSM]{Joint Statistical Meetings, Seattle, WA, Aug 6-9, 2006}

\begin{document}
\frame{\titlepage}

\frame{
\begin{abstract}
\setbeamerfont{small}{size=\tiny}
\usebeamerfont{small}
  
Open-Source statistical software is being used with increasing
frequency for the analysis of pharmaceutical data, particularly in
support of ``omics'' technologies within discovery. While it is
relatively straightforward to employ open-source tools for basic
research, software used in any regulatory context must
meet more rigorous requirements for documentation, training,
software life-cycle management, and technical support.

\vspace{1em}

We will focus on R, a full-featured open-source statistical
software package. We'll briefly outline the benefits it provides, as
seen from the perspective of a discovery statistician, show some
example areas in which it may be used, and then discuss the
documentation, training, and support required for this class of use.

\vspace{1em}

Next we will discuss what is needed for organizations to be
comfortable with employing open-source statistical software for
regulatory use within clinical, safety, or manufacturing. We will
then talk about how well or poorly R meets these requirements,
highlighting current issues.  Finally, we will discuss options for
third-party commercial support for R, and evaluate how well they
meet the requirements for use of R within both regulated and
non-regulated contexts.

\end{abstract}
}

\frame{
  \frametitle{Outline}
  \tableofcontents[pausesections] 

}

\section{Introduction}

\frame{
  \frametitle{Introduction}

  Open-Source statistical software is being used with increasing
  frequency for the analysis of pharmaceutical data, particularly in
  support of ``omics'' technologies within discovery. While it is
  relatively straightforward to employ open-source tools for basic
  research, software used in any regulatory context must meet more
  rigorous requirements for documentation, training, software
  life-cycle management, and technical support.
}

\section{Requirements}

\frame{
  \frametitle{Requirements}

  Software used in mission critical and regulated contexts must
  exhibit 7 key attributes:

  \begin{enumerate}
  \item Functional
  \item Verifiable
  \item Repeatable
  \item Documentable
  \item Auditable
  \item Stable
  \end{enumerate}
}

\frame{
\frametitle{Requirements (detail)}

\begin{description}[<+->]

\item[Functional] Performs the reqiured tasks

\item[Verifiable] Demonstrate that computer output is correct, or at least consustent..

\item[Repeatable] Given the same data, the same results can be
  obtained, potentially much later in time.

\item[Documentable] Documentation is avaiable or can easily be
  generated for the entire software lifecycle: Specification, Design,
  Development. Testing, Deployment, Change Management

\end{description}

}

\frame{
  \frametitle{Requirements (detail, cont)}

\begin{description}

\item[Auditable] Track everything done to data and the system

\item[Stable] Doesn't change too fast, so that there is enough time to
  develop required documentation

\item[Supported] Guaranteed (by \$\$) availability of external expense for
  \begin{itemize}
    \item installation
    \item problem resolution
    \item bug fixes
    \item feature development
    \item training
    \item application development
    \item consulting
  \end{itemize}

\end{description}
}

\section{What is R?}

\frame{
  \frametitle{What is R?}

  \begin{itemize}
    \item{foo}
  \end{itemize}
}


\section{ Status of R}

\frame{
  \frametitle{Status of R}

\begin{description}

\item[Functional] +++ This is R's strength, and is largely provided by
  user-supplied add on packages.  R currently provides more
  functionality than any other statistical software system and is
  growing rapidly.

\item[Verifiable] --- Most of the functionality of R comes from
  user-developed add-on packages, but there is currently no formal
  mechanism for evaluating the level of quality of these packages (eg:
  development, test, production, peer reviewed, validated) or
  documentation that they accomplish the required tasks.

\item[Repeatable] --- 

\item[Documentable] --- While the R core team has a well defined and
  managed process for design, development, testing, release, and
  change management, no formal documentation of this process appears
  to exists (aside from the specifications of the language itself).
  No centrally defined or managed process appears to exist for add-on
  packages.
\end{description}
}

\frame{

  \frametitle{Status of R (cont)}

\begin{description}

\item[Auditable] --- R has no built-in no audit log, either for data
  analysis steps or for changes to the system (e.g.: package updates,
  patches)

\item[Stable] --- The R core team releases minor (major.minor.patch)
  versions twice a year.  Since bug fixes are currently applied only
  to the latest released version of the system, it is difficult to
  properly support embedded and validated systems where one may need
  to resolve bugs in R, but must constrain the R version to remain
  constant for long periods due to the burden of documentation and
  testing that must be performed.

\item[Supported] While there is an increasingly large pool of
  statisticians and statistical consulting groups that have R
  expertise, no organization formally supports R at this time.

\end{description}
}

\section{ What needs to be done?}

\frame{
  \frametitle{Status of R (cont)}

}

\begin{frame}[allowframebreaks]
  \frametitle<presentation>{For Further Reading}
    
  \begin{thebibliography}{10}
    
  \beamertemplatebookbibitems
  % Start with overview books.

  \bibitem{Author1990}
    A.~Author.
    \newblock {\em Handbook of Everything}.
    \newblock Some Press, 1990.
 
    
  \beamertemplatearticlebibitems
  % Followed by interesting articles. Keep the list short. 

  \bibitem{Someone2000}
    S.~Someone.
    \newblock On this and that.
    \newblock {\em Journal of This and That}, 2(1):50--100,
    2000.
  \end{thebibliography}
\end{frame}


\end{document}
