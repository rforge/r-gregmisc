\documentclass[a4paper]{report}
\usepackage{Rnews}
\usepackage[round]{natbib}
\bibliographystyle{abbrvnat}

\begin{document}

\begin{article}

\title{Working with Unknown Values}
\subtitle{The \pkg{gdata} package}
\author{by Gregor Gorjanc}

\maketitle

This vignette has been published as \cite{Gorjanc}.

\section{Introduction}

Unknown or missing values can be represented in various ways. For example
SAS uses \code{.}~(dot), while \R{} uses \code{NA}, which we can read as
Not Available. When we import data into \R{}, say via \code{read.table} or
its derivatives, conversion of blank fields to \code{NA} (according to
\code{read.table} help) is done for \code{logical}, \code{integer},
\code{numeric} and \code{complex} classes. Additionally,
the \code{na.strings}
argument can be used to specify values that should also be converted to
\code{NA}. Inversely, there is an argument \code{na} in \code{write.table}
and its derivatives to define value that will replace \code{NA} in exported
data. There are also other ways to import/export data into \R{} as
described in the {\emph R Data Import/Export} manual \citep{RImportExportManual}.
However, all approaches lack the possibility to define unknown value(s) for
some particular column. It is possible that an unknown value in one column is a
valid value in another column. For example, I have seen many datasets where
values such as 0, -9, 999 and specific dates are used as column specific unknown
values.

This note describes a set of functions in package \pkg{gdata}\footnote{
package version 2.3.1} \citep{WarnesGdata}: \code{isUnknown},
\code{unknownToNA} and \code{NAToUnknown}, which can help with testing for
unknown values and conversions between unknown values and \code{NA}. All
three functions are generic (S3) and were tested (at the time of writing)
to work with: \code{integer}, \code{numeric}, \code{character},
\code{factor}, \code{Date}, \code{POSIXct}, \code{POSIXlt}, \code{list},
\code{data.frame} and \code{matrix} classes.

\section{Description with examples}

The following examples show simple usage of these functions on
\code{numeric} and \code{factor} classes, where value \code{0} (beside
\code{NA}) should be treated as an unknown value:

\begin{smallverbatim}
> library("gdata")
> xNum <- c(0, 6, 0, 7, 8, 9, NA)
> isUnknown(x=xNum)
[1] FALSE FALSE FALSE FALSE FALSE FALSE TRUE
\end{smallverbatim}

The default unknown value in \code{isUnknown} is \code{NA}, which means
that output is the same as \code{is.na} --- at least for atomic
classes. However, we can pass the argument \code{unknown} to define which
values should be treated as unknown:

\begin{smallverbatim}
> isUnknown(x=xNum, unknown=0)
[1] TRUE FALSE  TRUE FALSE FALSE FALSE FALSE
\end{smallverbatim}

This skipped \code{NA}, but we can get the expected answer after appropriately
adding \code{NA} into the argument \code{unknown}:

\begin{smallverbatim}
> isUnknown(x=xNum, unknown=c(0, NA))
[1] TRUE FALSE  TRUE FALSE FALSE FALSE TRUE
\end{smallverbatim}

Now, we can change all unknown values to \code{NA} with \code{unknownToNA}.
There is clearly no need to add \code{NA} here. This step is very handy
after importing data from an external source, where many different unknown
values might be used. Argument \code{warning=TRUE} can be used, if there is
a need to be warned about ``original'' \code{NA}s:

\begin{smallverbatim}
> xNum2 <- unknownToNA(x=xNum, unknown=0)
[1] NA  6 NA  7  8  9 NA
\end{smallverbatim}

Prior to export from \R{}, we might want to change unknown values (\code{NA}
in \R{}) to some other value. Function \code{NAToUnknown} can be used for
this:

\begin{smallverbatim}
> NAToUnknown(x=xNum2, unknown=999)
[1] 999   6 999   7   8   9 999
\end{smallverbatim}

Converting \code{NA} to a value that already exists in \code{x} issues an
error, but \code{force=TRUE} can be used to overcome this if
needed. But be warned that there is no way back from this step:

\begin{smallverbatim}
> NAToUnknown(x=xNum2, unknown=7, force=TRUE)
[1] 7 6 7 7 8 9 7
\end{smallverbatim}

Examples below show all peculiarities with class \code{factor}.
\code{unknownToNA} removes \code{unknown} value from levels and inversely
\code{NAToUnknown} adds it with a warning. Additionally, \code{"NA"} is
properly distinguished from \code{NA}. It can also be seen that the argument
\code{unknown} in functions \code{isUnknown} and \code{unknownToNA} need
not match the class of \code{x} (otherwise factor should be used) as the test
is internally done with \code{\%in\%}, which nicely resolves coercing
issues.

\begin{smallverbatim}
> xFac <- factor(c(0, "BA", "RA", "BA", NA, "NA"))
[1] 0    BA   RA   BA   <NA> NA
Levels: 0 BA NA RA
> isUnknown(x=xFac)
[1] FALSE FALSE FALSE FALSE  TRUE FALSE
> isUnknown(x=xFac, unknown=0)
[1]  TRUE FALSE FALSE FALSE FALSE FALSE
> isUnknown(x=xFac, unknown=c(0, NA))
[1]  TRUE FALSE FALSE FALSE  TRUE FALSE
> isUnknown(x=xFac, unknown=c(0, "NA"))
[1]  TRUE FALSE FALSE FALSE FALSE  TRUE
> isUnknown(x=xFac, unknown=c(0, "NA", NA))
[1]  TRUE FALSE FALSE FALSE  TRUE  TRUE

> xFac <- unknownToNA(x=xFac, unknown=0)
[1] <NA> BA   RA   BA   <NA> NA
Levels: BA NA RA
> xFac <- NAToUnknown(x=xFac, unknown=0)
[1] 0  BA RA BA 0  NA
Levels: 0 BA NA RA
Warning message:
new level is introduced: 0
\end{smallverbatim}

These two examples with classes \code{numeric} and \code{factor} are fairly simple and we
could get the same results with one or two lines of \R{} code. The real
benefit of the set of functions presented here is in \code{list} and
\code{data.frame} methods, where \code{data.frame} methods are merely
wrappers for \code{list} methods.

We need additional flexibility for \code{list}/\code{data.frame} methods,
due to possibly having multiple unknown values that can be different
among \code{list} components or \code{data.frame} columns. For these two
methods, the argument \code{unknown} can be either a \code{vector} or
\code{list}, both possibly named. Of course, greater flexibility (defining
multiple unknown values per component/column) can be achieved with
a \code{list}.

When a \code{vector}/\code{list} object passed to the argument \code{unknown} is not
named, the first value/component of a \code{vector}/\code{list} matches the first
component/column of a \code{list}/\code{data.frame}. This can be quite
error prone, especially with \code{vectors}. Therefore, I encourage the use of
a \code{list}. In case \code{vector}/\code{list} passed to argument
\code{unknown} is named, names are matched to names of \code{list} or
\code{data.frame}. If lengths of \code{unknown} and \code{list} or
\code{data.frame} do not match, recycling occurs.

The example below illustrates the application of the
described functions to a list which is
composed of previously defined and modified numeric (\code{xNum}) and
factor (\code{xFac}) classes. First, function \code{isUnknown} is used with
\code{0} as an unknown value. Note that we get \code{FALSE} for \code{NA}s as
has been the case in the first example.

\begin{smallverbatim}
> xList <- list(a=xNum, b=xFac)
$a
[1]  0  6  0  7  8  9 NA

$b
[1] 0  BA RA BA 0  NA
Levels: 0 BA NA RA
> isUnknown(x=xList, unknown=0)
$a
[1] TRUE FALSE  TRUE FALSE FALSE FALSE FALSE

$b
[1] TRUE FALSE FALSE FALSE  TRUE FALSE
\end{smallverbatim}

We need to add \code{NA} as an unknown value. However, we do not get the
expected result this way!

\begin{smallverbatim}
> isUnknown(x=xList, unknown=c(0, NA))
$a
[1] TRUE FALSE  TRUE FALSE FALSE FALSE FALSE

$b
[1] FALSE FALSE FALSE FALSE FALSE FALSE
\end{smallverbatim}

This is due to matching of values in the argument \code{unknown} and components
in a \code{list}; i.e., \code{0} is used for component \code{a} and \code{NA}
for component \code{b}.  Therefore, it is less error prone and more
flexible to pass a \code{list} (preferably a named list) to the argument
\code{unknown}, as shown below.

\begin{smallverbatim}
> xList1 <- unknownToNA(x=xList,
+                       unknown=list(b=c(0, "NA"), a=0))
$a
[1] NA  6 NA  7  8  9 NA

$b
[1] <NA> BA   RA   BA   <NA> <NA>
Levels: BA RA
\end{smallverbatim}

Changing \code{NA}s to some other value (only one per component/column) can
be accomplished as follows:

\begin{smallverbatim}
> NAToUnknown(x=xList1, unknown=list(b="no", a=0))
$a
[1] 0 6 0 7 8 9 0

$b
[1] no BA RA BA no no
Levels: BA no RA

Warning message:
new level is introduced: no
\end{smallverbatim}

A named component \code{.default} of a \code{list} passed to argument
\code{unknown} has a special meaning as it will match a component/column with
that name and any other not defined in \code{unknown}. As such it is very
useful if the number of components/columns with the same unknown value(s)
is large. Consider a wide \code{data.frame} named \code{df}. Now
\code{.default} can be used to define unknown value for several columns:

\begin{smallverbatim}
> df <- unknownToNA(x=df,
+                   unknown=(.default=0,
+                            col1=999,
+                            col2="unknown"))
\end{smallverbatim}

If there is a need to work only on some components/columns you can of
course ``skip'' columns with standard \R{} mechanisms, i.e.,
by subsetting \code{list} or \code{data.frame} objects:

\begin{smallverbatim}
> cols <- c("col1", "col2")
> df[, cols] <- unknownToNA(x=df[, cols],
+                           unknown=(col1=999,
+                                    col2="unknown"))
\end{smallverbatim}

\section{Summary}

Functions \code{isUnknown}, \code{unknownToNA} and \code{NAToUnknown}
provide a useful interface to work with various representations of
unknown/missing values. Their use is meant primarily for shaping the data
after importing to or before exporting from \R{}. I welcome any comments or
suggestions.

% \bibliography{refs}

\begin{thebibliography}{1}
\providecommand{\natexlab}[1]{#1}
\providecommand{\url}[1]{\texttt{#1}}
\expandafter\ifx\csname urlstyle\endcsname\relax
  \providecommand{\doi}[1]{doi: #1}\else
  \providecommand{\doi}{doi: \begingroup \urlstyle{rm}\Url}\fi

\bibitem[Gorjanc(2007)]{Gorjanc}
G.~Gorjanc.
\newblock Working with unknown values: the gdata package.
\newblock \emph{R News}, 7\penalty0 (1):\penalty0 24--26, 2007.
\newblock URL \url{http://CRAN.R-project.org/doc/Rnews/Rnews_2007-1.pdf}.

\bibitem[{R Development Core Team}(2006)]{RImportExportManual}
{R Development Core Team}.
\newblock \emph{R Data Import/Export}, 2006.
\newblock URL \url{http://cran.r-project.org/manuals.html}.
\newblock ISBN 3-900051-10-0.

\bibitem[Warnes (2006)]{WarnesGdata}
G.~R. Warnes.
\newblock \emph{gdata: Various R programming tools for data manipulation},
  2006.
\newblock URL
  \url{http://cran.r-project.org/src/contrib/Descriptions/gdata.html}.
\newblock R package version 2.3.1. Includes R source code and/or documentation
  contributed by Ben Bolker, Gregor Gorjanc and Thomas Lumley.

\end{thebibliography}

\address{Gregor Gorjanc\\
  University of Ljubljana, Slovenia\\
\email{gregor.gorjanc@bfro.uni-lj.si}}

\end{article}

\end{document}
