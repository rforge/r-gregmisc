\documentclass[12pt]{article}
\usepackage{geometry}

\title{Software Development Site for R packages\\
       Version $Id$ }
\author{Gregory R. Warnes\\
  Associate Director\\
  Groton Nonclinical and Clinical Sciences Statistics\\
  Pfizer Global Research and Development}

\begin{document}

\maketitle

\section{Introduction}

The R statistical software environments are increasingly being used
for data analysis within Pfizer.  A major attraction of this system is
the relative ease with which add-on packages can be developed and
deployed, as well as the extensive set of packages that are already
available.  (At the time of this writing, more than 400 add-on
packages already available.)  As a consequence, add-on packages
implementing advanced statistical methods often appear concurrent
with--or even before--the publication of the manuscripts describing
the methods themselves.  The combination of these factors has placed R
as a premier tool for statistical computation in many quickly
developing areas, including bioinformatics and systems biology.

The core R environment, under the guidance of the R Core Team, is
developed using a standard software model supported by software
development tools\footnote{These tools include a source code version
  control system, an issue-tracking system, file release area,
  developer and user mailing lits, and web documentation.}  hosted by
the R Project web sites (http://www.r-project.org).  This makes it a
relatively straightforward task to validate the use of the R itself
for regulatory environtments.  In fact, Tony Rossiny of Novartis is
leading an effort to provide the necessary documentation to make this
possible.

Unfortunately, the R project does not currently supply equivalent
software deveolopment tools for authors of user contributed packages.
This leaves the authors of these packages to use whatever software
development tools they personally have access to.  

While the R project requires contributed package to passes a minimal
set of tests that ensure proper package structure and
consistency\footnote{This set of tests will automatically perform
  additional regression and functionality tests if appropriate code is
  in place.  However, such code is not required.}, there is no
mechanism for categorizing the quality of these add on packages.  As a
consequence, the quality of add-on R packages is extremely varied and
there is no straightforward mechanism for determining package quality.
This makes it extremely difficult to validate the use of add-on R
packages for use in a regulated environment.

This project proposes to develop an internet portal that provides
add-on package developers with the same types of tools that are
available to the developers of the core R environment, and to
institute a system for clearly categorizing packages according to
thier development status.  This will encourage package developers to
utilize a standard software lifecycle, which will increase the overall
quality of contributed R packages, will make it easier to determine
which package may be appropriately used in specific circumstances, and
will ultimately make it possible to validate a set of well-supported
packages for use in regulated environments.


\section{Goals}

This project has three goals:

\begin{enumerate}
\item Enable and encourage the use of an appropriate software
  development lifecycle for user-contributed R add-on packages.
   
\item Reward package authors with publication credit for the
  development of high-quality add-on packages.
   
\item Provide a set of well-designed packages with a documented
  software lifecycle process that can easily be validated for use in
  regulated environments.

\end{enumerate}

\section{Benefits}

Providing an appropriate set of software lifecycle tools to R packages
developers will:
\begin{enumerate}
\item Encourage the use of appropriate software lifecycle methods to R
  packages
\item Allow users to easily evaluate the development status of
  available packages.
\item Encourage the development of high-quality add on packages
\item Reduce the risk in using add-on R packages
\item Provide a set of packages which can be easily validated for use
  in regulated environments (GxP, FDA Part 11 compliance, etc.)
\end{enumerate}

\section{Mechanism}

Create and support an internet portal for add-on package developers
that provides software lifecycle tools.  These tools include
\begin{itemize}
\item version control system (cvs/subversion)
\item file release area 
\item web page
\item news lists
\item issue tracking system
\end{itemize} for each individual package.

This portal will support categorizing software releases as 
\begin{itemize}
\item development,
\item testing,
\item production, or
\item peer-reviewed/validated.
\end{itemize}

Appropriate qualifications for each release classification will be
established.  Examples of such qualifications include: 
\begin{itemize}
\item All package releases must pass the 'R CMD check'
  standard R package tests.
\item Test packages must include working examples for each documented
  function.  These will be used for basic regression testing.
\item Production packages must include a 'vignette' describing the
  basic features of the package and show how these features are used
  in a real analysis.
\item Peer-reviewed/validated packages must include a reasonably
  complete set of unit tests, and must be subjected to code review by
  two independent individuals.
\end{itemize}

In order to encourage academic package developers to submit thier
packages to the peer-review/validation process, packages that are
accepted into this category will become publications in the
\emph{Journal of Statistical Software}, a well recognized
peer-reviewed journal.

In order to obtain a sufficent pool of software reviewers, each
individual submitting a package to the peer review process will be
required to commit to reviewing (or arranging for reviews) of two
other software packages.  Alternatively, individuals may elect to pay
a fee which will be used to renumerate reviewers for timely effort.

Software reviewers will complete a standard review form developed by
the community which will ensure that best practices are followed by
both software authors and by reviewers.  This form will include a
commentary on the software that will be published alongside the
software itself once all revisions have been completed and the package
has been accepted for publication. 

\section{Resources}

\subsection{Initial Site Creation}

\begin{itemize}
\item Initial Design Specification - 1 man month

\item Programming - 3 man months
  
  This process should be a straightforward extension of existing
  sourceforge-style tools.  One particularly attractive option is
  'GForge', available from http://gforge.org and as a debian package.

  This will include 
  \begin{itemize}
  \item installation and configuration of the web server
  \item design of the web interface and project templates
  \item import of existing R packages
  \item integration with the existing R project web site & tools
  \item security evaluation and system hardening
  \end{itemize}
    
\item User Acceptance Testing - 3 months?

\item Train site maintainer - 3 month?  (concurrent with acceptance testing?)

\end{itemize}

\subsection{Ongoing Maintenace}

\begin{itemize}

\item Web site hosting \$500/yr  (guess)

\item Web site administrator and maintainer - 20hrs/week (probably and
  over-estimage)

  This individual will:
  
  \begin{enumerate}

  \item Ensure the ongoing operation of the web site, include host and
    site adminstration (regular backups!).

  \item Provide technical support for package developers.
  
  \item Perform some ongoing development to improved the available
    tools, site documention, category standards, etc.
  
  \item Acting as a gatekeeper for category changes, including
    assignment of reviewers, processing of review responses, etc.

  \end{enumerate}

\end{itemize}


\end{document}