\documentclass{report}
\usepackage{Rnews}
\begin{document}

\author{by Gregory R. Warnes}
\title{The gregmisc package: something for everyone}
\subtitle{}

\maketitle

The gregmisc package is a repository for functions that I find myself
needing but that don't seem to be available somewhere else.  In
contrast to many packages, it is not organized thematically, rather
it contains a hodge-podge of different routines that make my life a
little easer.

I've taken the time to bundle these functions up as a package for
three reasons, first to share my efforts with others, second to get
feedback (particularly bug reports and feature enhancements), and
third to force myself to properly document and test the code.

The function in the gregmisc library fall into six general areas:
permutations and combinations, tools for linear models, plots, data
manipulation, fixed or extended versions of existing functions, and other.

\begin{enumerate}


\item{permutation and combinations}

\begin{description}
\item[combinations]{ Enumerate the combinations of the elements of a vector}
\item[permutations]{ Enumerate the permutations of the elements of a vector}
\item[factorial]{ Compute the factorial function}
\end{description}

\item{tools for linear models}

\begin{description}
\item[ci]{ Compute confidence intervals}
\item[contrast.lm]{ Compute (and test) arbitrary single-term     
        contrasts for regression objects}       
\item[estimable]{ Compute and test estimable linear functions of
         the fitted coefficients (including contrasts) of regression objects}
\item[glh.test]{ Test a general linear hypothesis for regression objects}

\end{description}

\item{plots}
\begin{description}
\item[boxplot.n]{Produce a boxplot annotated with the number of observations}
\item[plotCI]{ Plot error bars}
\item[plotmeans]{ Plot group means and confidence intervals}
\item[wapply]{ Compute the value of a function over a local region of
    an x-y plot}
\item[space]{ Deterministically space points in an x-y plot so they don't
 overlap.}
\item[hist2d]{ Create and Plot a 2-dimensional histogram.}
\item[bandplot] {Plot x-y points with locally smoothed mean and standard deviation}

\end{description}

\item{data manipulation}

\begin{description}
\item[combine]{ Combine R objects (such as dataframes) and add an
    additional column labeling the source}
\item[rename.vars]{ Rename variables in a dataframe}
\end{description}

\item{fixed or extended versions of current functions}

\begin{description}
\item[lowess]{ Extend built-in \verb+lowess+ function to handle model
    formulae }
\item[plot.lm]{ Extend the built-in \verb+plot.lm+ function: }
    \begin{itemize}
    \item residual plots: add rug, zero line, mean and 1-sigma smooths 
    \item residual quantile plots: add \verb+qqline+
    \item add plots of each predictor against the residuals.
    \end{itemize}
\end{description}

\item{other}

\begin{description}
\item[running]{ Apply a function over adjacent subsets of a vector}
\end{description}

\end{enumerate}

I welcome comments and contributions.  The current package includes
code by Ben Bolker, Bendix Carstensen, Don MacQueen, and William
Venables.


\address{Gregory R. Warnes \\
        Pfizer Global Research and Development \\
        \emph{gregory\_r\_warnes$@$groton.pfizer.com} }  %%!!!%%


\end{multicols}  %%!!!%%

\end{document}
